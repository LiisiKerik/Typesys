\documentclass[runningheads]{llncs}
\usepackage{amsmath}
\usepackage{array,color,listings}
\usepackage[T1]{fontenc}
% \usepackage[showframe]{geometry}
\usepackage{longtable}
\frenchspacing
\newcolumntype{L}{>{$}l<{$}}
\newcommand\kood[1]{\text{\color{red}\texttt{#1}}}
\newcommand\nurksulud[1]{\langle\text{#1}\rangle}

\lstset{language=haskell,mathescape}
\begin{document}
%
\title{Typeclasses as subkinds}
%
\author{Liisi Kerik%\inst{1} 
\and  Kalmer Apinis
\and  Härmel Nestra
}
%
\authorrunning{L. Kerik et al.}
%
\institute{
Deptartment of Computer Science, University of Tartu, \\
Narva mnt 18, EE-51009 Tartu, Estonia \\
\email{\{liisi.kerik,kalmer.apinis,harmel.nestra\}@ut.ee}}
%
\maketitle             
%
\begin{abstract}
\keywords{First keyword  \and Second keyword \and Another keyword.}
\end{abstract}
%
%
\section{Introduction}
% Sektsioonid on esialgsed -- hiljem paneme neid kokku jne.
% Kirjuta minimalistlikult kuid täielikult. Kirjuta nagu geeniusest magistrandile, kes saab kõigest kiiresti aru aga ei tea konkreetsest probleemist midagi.
\section{Typeclasses}
%Siia kirjutada sellest, mis on tüübiklassid%

In Haskell 98\cite{jones1999report} you can declare a typeclass and its operators using this general form:
\begin{lstlisting}
  class $\textit{cx}$ => C u where $\textit{cdexls}$ 
\end{lstlisting}
It creates a typeclass $C$ with a type parameter $u$. The type parameter can be used in the superclass declaration $\textit{cx}$ and method signatures $\textit{cdexls}$. It is not allowed to declare typeclasses that would create cycles in the superclass relation. Class declarations $\textit{cdexls}$ contains method declarations $v_i$ in the form of explicit type signatures
\begin{lstlisting}
  $v_i$ :: $\textit{cx}_i$ => $t_i$
\end{lstlisting}
In Haskell, class declarations also may contain fixity declarations and default class methods---we omit these for simplicity.

For example, the equality typeclass can be declared as such:
\begin{lstlisting}
  class Eq a where eq :: a -> a -> Bool
\end{lstlisting}

% ...
% edasi: operaator eq: Eq a => a -> a -> Bool
% edasi: instantsid
% edasi: Eq näide 
% edasi: Eq-d kasutava avaldise näide 


\section{Kinds and Subkinds}

% mis on liigid 
% millele me siin viitame? Vaadata "Giving Haskell a Promotion" viiteid
% liigi näide: * -> *, kui vaja, mainida et seda on vaja listide andmestruktuuri juures
% mis on alamliigid
% näide???

\section{Typeclasses as Subkinds}

% Näite põhjal aga siis ka üldisemad reeglid

\section{Beyond Haskell 98 -- multi-parameter}

\section{Beyond Haskell 98 -- categoical view??}


\bibliographystyle{splncs04}
\bibliography{main}

\end{document}